\section{Strategia risolutiva e Architettura proposta}
Per risolvere il problema è stato utilizzato il framework fornito da java.util.concurrent, che mette a disposizione un \textbf{ExecutorService}; tale interfaccia espone delle API per eseguire computazioni concorrenti in modo standardizzato così da non dover ricorrere a meccanismi di basso livello.\newline
Tramite la factory \textit{Executors} è possibile creare istanze di vari ExecutorService. Nel caso in esame è stato utilizzato un FixedThreadPool con un numero di thread pari al numero di processori presenti sulla macchina che esegue il programma.
\subsection{Decomposizione in Task}
Nella soluzione scelta è stato individuato un unico Task:
\begin{itemize}
    \item \textbf{ComputeDocumentTask} \newline
    Questo task si occupa di caricare in memoria centrale un documento. Una volta estrapolato il contenuto come stringa si occupa di analizzare le singole parole. I risultati parziali del conteggio verranno salvati in una struttura dati locale, la quale verrà restituita dal task come risultato.
\end{itemize}

\noindent I ComputeDocumentTask estendono l'interfaccia \textbf{Callable}. L'ExecutorService si occupa di eseguire il metodo \textit{call} di ogni Callable che gli viene fornito, ovvero uno per ogni documento che dovrà essere computato. L'ExecutorService non accetterà poi più nuovi task da eseguire e si metterà in attesa che quelli in esecuzione terminino.\newline
Infine, il risultato di ogni ComputeDocumentTask viene utilizzato per aggiornare la struttura dati centrale, che fornisce il risultato finale della computazione.\newline

\noindent Ogni ComputeDocumentTask opera in maniera indipendente rispetto agli altri per tutta la durata del suo ciclo di vita. Al termine dell'esecuzione di tutti i task, le strutture dati vengono unite in modo sequenziale, non richiedendo alcun meccanismo di sincronizzazione.\newline

\noindent La GUI permette all'utente di avviare la computazione e di fermarla in qualunque momento; per fermare l'ExecutorService è sufficiente utilizzare il metodo \textit{shutdownNow()}, il quale termina nel più breve tempo possibile l'esecuzione di tutti i task.\newline

% Descrizione della strategia risolutiva e dell’architettura proposta, sempre
% focalizzando l’attenzione su aspetti relativi alla concorrenza.
